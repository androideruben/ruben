%%%%%%%%%%%%%%%%%%%%%%%%%%%%%%%%%%%%%%%%%%%%%%%%%%%%%%%%%%%%%%%%%%%%%%%%%%%%%%%%%%%%%%%%%%%%%%%%%%%%%%%%%%%%%%%%%%%%%%
\documentclass[mathserif]{beamer} 
\usepackage{graphicx,hyperref,fdactpstats,url,verbatim,fancyvrb,bm}
\def \red{\textcolor{red}}
\def \red{\textcolor{blue}}

%%%%%%%%%%%%%%%%%%%%%%%%%%%%%%%%%%%%%%%%%%%%%%%%%%%%%%%%%%%%%%%%%%%%%%%%%%%%%%%%%%%%%%%%%%%%%%%%%%%%%%%%%%%%%%%%%%%%%%
\begin{document}

	\title{Linear Mixed Models}  
	\author{Ruben Montes de Oca, MS.}
	\institute{FDA/CTP/Statistics Branch I}
	\date{October, 2018} 

%%%%%%%%%%%%%%%%%%%%%%%%%%%%%%%%%%%%%%%%%%%%%%%%%%%%%%%%%%%%%%%%%%
\frame{\titlepage} 
%%%%%%%%%%%%%%%%%%%%%%%%%%%%%%%%%%%%%%%%%%%%%%%%%%%%%%%%%%%%%%%%%%
%\section{INTRODUCTION}
\begin{frame}
	\frametitle{INTRODUCTION}
	
There are a variety of models for analyzing correlated data, among them Linear Mixed Models (LMM) and 
Generalized Estimating Equations (GEE) are two of the most popular that we find reviewing clinical trials. \\ \\

The LMM are part of a broader class called Generalized Linear Models (GLM) where the response variable 
distribution is not necessarily normal. \\ \\
	
LMM models are appealing because of their:

\begin{itemize}
\item flexibility for analyzing a variety of study designs
\item flexibility of dealing with imbalance due to missing data
\item ability to parsimoniously model the variance and correlation
\item ability to predict individual trajectories over time
\end{itemize}


\end{frame}

%%%%%%%%%%%%%%%%%%%%%%%%%%%%%%%%%%%%%%%%%%%%%%%%%%%%%%%%%%%%%%%%%%
\section{MATHEMATICAL FORMULATION}
\begin{frame}
	\frametitle{MATHEMATICAL FORMULATION}
	
The LMM is defined by	

	\begin{equation} 
	\begin{gathered}
	\textbf{Y}= \bm{X} \bm{\beta_{1}} + \bm{Z} \bm{u} + \bm{\varepsilon}
	\end{gathered}
	\end{equation}

and 

\item \bm{u} \sim \mathcal{N}(\bm{0,G}), \bm{\varepsilon} \sim \mathcal{N}(\bm{0,R})

\vspace{3}

The \textit{Fixed Factors} $\bm{X}$ are categorical variables that include all levels of interest in the study and the 
\textit{Fixed Effects} are the $\bm{\beta}$ coefficients.

\vspace{3}

The Random Factors $\bm{Z}$ are categorical variables with levels that are randomly sampled from a population under study and 
the Random Effects are the $\bm{u}$ coefficients.

	 \end{frame}



%%%%%%%%%%%%%%%%%%%%%%%%%%%%%%%%%%%%%%%%%%%%%%%%%%%%%%%%%%%%%%%%%%
\section{EXAMPLE}
\begin{frame}
	\frametitle{EXAMPLE: HAPPINES AND MEDICATION OVER TIME}
		
%\begin{center}
 \begin{tabular}{||c c c c c||} 
 \hline
Id & Gender & Time & HappyScore & DosisMed \\ [0.5ex] 
 \hline\hline
1 & 1 & 0 & 20 & 1000 \\ [0.5ex] 
1 & 1 & 2 & 18 & 1100 \\
1 & 1 & 3 & 15 & 1200 \\
1 & 1 & 6 & 20 & 1300 \\
2 & 2 & 0 & 22 & 1000 \\
\vdots & \vdots & \vdots & \vdots & \vdots \\
6 & 2 & 2 & 28 & 1100 \\
6 & 2 & 3 & 26 & 1109 \\
6 & 2 & 6 & 14 & 1500 \\ [1ex] 
\hline
\end{tabular}
%\end{center}

\end{frame}
	
%%%%%%%%%%%%%%%%%%%%%%%%%%%%%%%%%%%%%%%%%%%%%%%%%%%%%%%%%%%%%%%%%%
%\section{EXAMPLE}
\begin{frame}
	%\frametitle{EXAMPLE}
		
\begin{figure}
\centering
\includegraphics[scale=0.40]{GEE.png}

\includegraphics[scale=0.50]{LMM.png}
\end{figure}

\end{frame}
	
%%%%%%%%%%%%%%%%%%%%%%%%%%%%%%%%%%%%%%%%%%%%%%%%%%%%%%%%%%%%%%%%%%
\section{REFERENCES}
\begin{frame}
	\frametitle{REFERENCES}
		
\begin{itemization}

\item Weighted Methods for Analyzing Missing Data with the GEE Procedure \\
{http://support.sas.com/resources/papers/proceedings14/SAS166-2014.pdf}

\item To GEE or Not to GEE Comparing Population Average and Mixed Models for Estimating the Associations Between Neighborhood Risk Factors 
and Health\\
	\url{https://www.researchgate.net/publication/41895248_To_GEE_or_Not_to_GEE_Comparing_Population_Average_and_Mixed_Models_for_Estimating_
	the_Associations_Between_Neighborhood_Risk_Factors_and_Health}
	
\item Generalized Estimating Equations in Longitudinal Data Analysis: A Review and Recent Developments\\
		\url{http://downloads.hindawi.com/archive/2014/303728.pdf}

\item Applied Longitudinal Analysis, 2nd Edition by Garret Fitzmaurice et al.\\
	Lecture slides in black and white (BIO226) at\\
	\url{https://content.sph.harvard.edu/fitzmaur/ala2e/} 
\end{itemization}
	
\end{frame}
	
%%%%%%%%%%%%%%%%%%%%%%%%%%%%%%%%%%%%%%%%%%%%%%%%%%%%%%%%%%%%%%%%%%
%\section{REFERENCES}
\begin{frame}
	%\frametitle{REFERENCES}
		
\begin{itemization}

\item Marginal or conditional regression models for correlated non-normal data?\\
	\url{https://besjournals.onlinelibrary.wiley.com/doi/abs/10.1111/2041-210X.12623}



\item Analysis of long series of longitudinal ordinal data using marginalized models\\
	\url{https://www.sciencedirect.com/science/article/pii/S016794731500167X}
	
\item Sandwich estimator\\
		\url{https://www.stat.berkeley.edu/~census/mlesan.pdf}


\end{itemization}

\end{frame}
	
%%%%%%%%%%%%%%%%%%%%%%%%%%%%%%%%%%%%%%%%%%%%%%%%%%%%%%%%%%%%%%%%%%%%%%%%%%%%%%%%%%%%%%%%%%%%%%%%%%%%%%%%%%%%%%%%%%%%%%	
\end{document}
%%%%%%%%%%%%%%%%%%%%%%%%%%%%%%%%%%%%%%%%%%%%%%%%%%%%%%%%%%%%%%%%%%%%%%%%%%%%%%%%%%%%%%%%%%%%%%%%%%%%%%%%%%%%%%%%%%%%%%

