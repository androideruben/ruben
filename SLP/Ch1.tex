%%%%%%%%%%%%%%%%%%%%%%%%%%%%%%%%%%%%%%%%%%%%%%%%%%%%%%%%%%%%%%%%%%%%%%%%%%%%%%%%%%%%%%%%%%%%%%%%%%%%%%%%%%%%%%%%%%%%%%%%%%%%%%%%%%%
\documentclass[letterpaper,11pt]{article}

\usepackage{graphicx}
\usepackage{amsmath}
\usepackage{amssymb}
\usepackage{bm}         
\usepackage{amsthm}
\usepackage{afterpage}
\usepackage[table, usenames, dvipsnames]{xcolor} %added by CDE
\usepackage{latexsym}
\usepackage[framemethod=0,ntheorem]{mdframed}
\usepackage{multirow}
\usepackage{fullpage}
\usepackage{setspace}
\usepackage{multirow}
\usepackage{rotating}
\usepackage{pdflscape}
\usepackage{fancyhdr}
\usepackage[utf8]{inputenc}
\usepackage{lscape}
\usepackage{color}
\usepackage{tabularx}
\usepackage{array}
\usepackage{longtable}
\usepackage[english]{babel}
\usepackage{csquotes}
\usepackage[font=small,labelfont=bf,skip=0pt]{caption}
\captionsetup[table]{skip=0.2pt}
\captionsetup[table]{aboveskip=0pt}
\captionsetup[table]{belowskip=-15pt}

\usepackage{subcaption}

\captionsetup[subtable]{skip=2pt}

\usepackage[tableposition=top]{caption}
\usepackage[toc,page]{appendix}
\usepackage{amsfonts}

\setcounter{MaxMatrixCols}{30}
\providecommand{\U}[1]{\protect\rule{.1in}{.1in}}

\newcommand{\tablefont}{\fontsize{3mm}{3mm}\selectfont}
\MakeOuterQuote{"}

\newcommand{\MC}{\multicolumn}
\newcommand{\MR}{\multirow}

\headheight 15pt
\headsep 2em

\renewcommand{\footrulewidth}{0pt}
\newtheorem{remark}{Remark}
\newtheorem{comment}{\textbf{Comment}}
\newtheorem{definition}{Definition}
\newtheorem{discussion}{Discussion:}
\newtheorem{claim}{Claim}
\newtheorem{question}{Question}
\newtheorem{answer}{Answer}
\newcolumntype{T}{>{\tiny}l}
\newcolumntype{M}{>{\centering\arraybackslash}m{3.1cm}}

\usepackage{fancyhdr}
\fancyhf{}
\pagestyle{fancy} %added by CDE to allow headers, footers to work
%\rhead{Share\LaTeX}
%\lhead{Guides and tutorials}


\usepackage[pdftex]{hyperref}   
\hypersetup{colorlinks, citecolor=Violet, linkcolor=Mahogany, urlcolor=blue}

\rfoot{Page \thepage}
\lfoot{DRAFT-DELIBERATIVE-CONFIDENTIAL}  

%%%%%%%%%%%%%%%%%%%%%%%%%%%%%%%%%%%%%%%%%%%%%%%%%%%%%%%%%%%%%%%%%%%%%%%%%%%%%%%%%%%%%%%%%%%%%%%%%%%%%%%%%%%%%%%%%%%%%%%%%%%%%%%%%%%
\begin{document}
%%%%%%%%%%%%%%%%%%%%%%%%%%%%%%%%%%%%%%%%%%%%%%%%%%%%%%%%%%%%%%%%%%%%%%%%%%%%%%%%%%%%%%%%%%%%%%%%%%%%%%%%%%%%%%%%%%%%%%%%%%%%%%%%%%%

\vspace{-25pt}%
\begin{tabular}[t]{lp{1in}l}
	\multirow{5}{*}{\includegraphics[width = 2in]{fdalogo.png}} &  & U.S. Department of Health and Human Services\\
																								&  & Food and Drug Administration \\
																								&  & Center for Tobacco Products \\
																								&  & Office of Sciences \\
																								&  & Division of Population Health Science \\
																								&  & Statistics Branch \\
\end{tabular}

\newline\vspace{15pt} \newline{\Large \textsc{Applied Regression Analysis}}\newline\vspace{0.015in}

\begin{tabular}[h!]{p{2in} p{10in}}
	\rule{0pt}{4ex}\textbf{Reporting to:}          & Antonio Paredes, MA, MS  \\
																							   & Lead Mathematical Statistician \\
																							   & Division of Population Health Science \\
                                                 & Statistics Branch \\
																							   & \\
	\rule{0pt}{4ex}\textbf{Subject:}               & Applied Regression \\
	\rule{0pt}{4ex}\textbf{Date:}                  & \today \\
	\rule{0pt}{4ex}\textbf{Statistical Reviewer:}  & Ruben Montes de Oca, MS \\
																							   & Mathematical Statistician \\
																							   & Division of Population Health Science \\
                                                 & Statistics Branch \\
																							   & \\
	\rule{0pt}{4ex}\textbf{Branch Chief:}          & George Rochester, PhD \\
																							   & Chief, Statistics Branch \\
																							   & Division of Population Health Science \\
                                                 & Statistics Branch \\
                                                 & \\
	\rule{0pt}{4ex}\textbf{Key Words:}  					 & Regression \\
 \mbox{$\quad$} \\
 \mbox{$\quad$} \\
\end{tabular}

\newpage
\noindent 

%%%%%%%%%%%%%%%%%%%%%%%%%%%%%%%%%%%%%%%%%%%%%%%%%%%%%%%%%%%%%%%%%%%%%%%%%%%%%%%%%%%%%%%%%%%%%%%%%%%%%%%%%%%%%%%%%%%%%%%%%%%%%%%%%%%

%\begin{enumerate}
%\item
	%Null: $H_{0}: \beta_{9} \leq$ 0
%\item
	%Alternative: $\beta_{9} >$ 0
%\end{enumerate}


%\begin{eqnarray*}\label{eq:eqn4}
%E\left(Y_{ij} \mid Arm = 1, Timej= 1, \right) - E\left(Y_{ij} \mid Arm = 0, Timej= 1 \right) = \beta_{1} + \beta_{b_{j}}
%\end{eqnarray}

%$Y_{i}$ $\sim N(X_{i} \beta, Z_{i} D_{u} Z_{i}^{'} + R)$

%$Y_{i}= X_{i} \beta + Z_{i} u_{i} + \epsilon_{i}$ 

%\begin{equation}
%    ARM=
%    \begin{cases}
%      1, & \text{RNC}\\
%      0, & \text{UNC}
%    \end{cases}
%  \end{equation}

%page 2: %%%%%%%%%%%%%%%%%%%%%%%%%%%%%%%%%%%%%%%%%%%%%%%%%%%%%%%%%%%%%%%%%%%%%%%%%%%%%%%%%%%%%%%%%%%%%%%%%%%%%%%%%%%%%%%%%%%%%%%%%%%%%%%%%%%
\section*{REVIEW OF SIMPLE REGRESSION}

\section{LINEAR MODELS IN ONE VARIABLE}

	We will work on the simplest linear model:

	\begin{equation} 
	\begin{gathered}
	$Y_{i}= \beta_{0} + \beta_{1} X_{i}  + \epsilon_{i}$ 
	\end{gathered}
	\end{equation}

	\noindent 
	A straight line where the i indicates the observation units $Y_{i}$ with i=1, 2, ..., n, the $X_{i}$ are n observations of known 	
	constants, $\beta_{0}$ is the intercept, $\beta_{1}$ the slope and the $\epsilon_{i} \sim \mathcal{N}(0,\sigma^{2})$ independent 
	pairwise.  
\\ \\
	The observations of the dependent variable $Y_{i}$ are assumed to be random observations from the population of random variables with the 
	mean of each population given by E($Y_{i}$), the expected value of $Y_{i}$.

\section{LEAST SQUARES ESTIMATION}

	The two parameters $\beta_{0}$ and $\beta_{1}$ are estimated from the data and we will use least squares estimation; for the least square 
	estimation, we calculate the sum of the squared distances between a theoretical line and the observed points ($X_{i}, 
	Y_{i}$) in a plane. If we modify this theoretical line with intercept $\beta_{0}$ and slope $\beta_{1}$ we will have different sum of 
	squares. For all of these theoretical lines, we can find values for $\beta_{0}$ and $\beta_{1}$ that minimize the sum of squares, and 
	then, these $\beta$ estimates $\hat{\beta_{0}}$ and $\hat{\beta_{1}}$ are known as the least squares estimation.  
	
	The \textbf{Normal Equations} are:

	\begin{equation}
	\sum{Y_{i}}= n(\hat{\beta_{0}}) + (\sum{X_{i}) \hat{\beta_{1}}
	\end{equation}

	\begin{equation}
	\sum{X_{i} Y_{i}}= (\sum{X_{i}) \hat{\beta_{0}} + (\sum{X_{i}^2) \hat{\beta_{1}}
	\end{equation}

	the solution of these normal equations are the estimates of $\beta_{0}$ and $\beta_{1}$ obtained by least squares:
	
	\begin{equation}
	\hat{\beta_{1}}= \sum x_{i} y_{i} - \frac {\sum x_{i} y_{i}}  {\sum x_{i}^2 - ((1/n)*\sum x_{i})^2}
	\end{equation}

	\begin{equation}
	\hat{\beta_{0}}= \overline{Y} - \hat{\beta_{1}} \overline{X}
	\end{equation}
	

	Then, using these estimates our regression equation, also known as fitted regression line is:

	\begin{equation}
	\hat{Y{i}}= \hat{\beta_{0}} + \hat{\beta_{1}} X_{i}
	\end{equation}
	
\section{PREDICTED VALUES AND RESIDUALS}

	Each quantity computed from the fitted regression line $\hat{Y_{i}}$ is the estimate of the population mean of Y for a particular value of 
	X, and is also a prediction of the value of Y one might obtain on some future observation at the level of X, and the residuals 
	
	\begin{equation}
	e_{i}= Y_{i} - \hat{Y_{i}}
	\end{equation}
	
	\begin{equation}
	SS(Res)= \sum(e_{i})^2
	\end{equation}
	
	measure the discrepancy between the data and the fitted model.
	
\section{ANALYSIS OF VARIATION IN THE DEPENDENT VARIABLE}

	The residuals defined in equation (7) can be written also in this way:
	
	\begin{equation}	
	\begin{gathered}
	$Y_{i}= \hat{Y_{i}} + e_{i}$
	\end{gathered}	
	\end{equation}
		
	Then $\sum (Y_{i}^2)= \sum (\hat{Y_{i}}+ e_{i})^2= \sum (\hat{Y_{i}}^2) + \sum (e_{i})^2= SS(Model)+ SS(Res)$
	\\ \\
	(the term $\sum (\hat{Y_{i}} e_{i}$) is zero, see chapter 3, or derive with respect $\beta_{0}$, which is equal to zero because of 
	critical point for finding the minimum.)
	\\ \\
	The term SS(Model) is the sum of squares accounted for by the model and SS(Res) is the unaccounted part, and the SS($Total_{uncorr}$)=
	SS(Model)- SS(Res) is the uncorrected total sum of squares.
	\\ \\
	The more convenient formulas for computation are:
	
	\begin{equation}	
	SS(Model)= n \overline{Y}^2 + \hat{\beta_{1}}^2 \sum(X_{i} - \overline{X})^2
	\end{equation}
		
	\begin{equation}		
	SS(Res)= SS(Total_{uncorr})- SS(Model)
	\end{equation}	
	
	There are also the corrected sum of squares:
	\\ \\
	SS($Total_{corrected})= SS(Total_{uncorr}) - n \overline{Y}^2= ( SS(Model)- n \overline{Y}^2) + SS(Res)$ 
	
	\\ \\
	or using equations (10):
	\\ \\
	
		\begin{equation}
		\sum{y_{i}^2}= \hat{\beta_{1}} \sum (X_{i}- \overline{X})^2 + \sum{e_{i}} = SS(Reg) + SS(Res)
		\end{equation}
	
	\\ \\
	Notice that the lower case y is the deviation of Y from $\overline{Y}$ so that the $\sum y_{i}^2$ is the corrected sum of squares.
	
	One measure of contribution of the independent variable or variables in the model is the coefficient of determination, denoted by R^2:

	\begin{equation}	
	\begin{gathered}
	R^2= SS(Reg)/ \sum(y_{i})
	\end{gathered}
	\end{equation}
	
	This is the proportion of the (corrected) sum of squares of Y attributable to the information obtained from the independent variable or 
	variables. The interpretation is the proportion of variation in the dependent variable 
	explained by its linear relationship with the independent variable. It is a number between 0 and 1, and the closer to 1 the greater this 
	variation is explained. 

\section{TESTS OF SIGNIFICANCE AND CONFIDENCE INTERVALS}
	The most common hypothesis in simple linear regression is $H_{0}: \beta_{1}$= m, where m is any constant of interest.
	The alternative hypothesis is $H_{a}: \beta_{1} \neq m$, or $H_{a}: \beta_{1} \geq m$, or $H_{a}: \beta_{1} < m$.
	
	Since the errors $\epsilon_{i}$ are normally distributed, the $\hat{\beta_{1}}$ is normally distributed with mean $\beta_{1}$ and variance 
	Var($\hat{\beta_{1}}$).
	\\ \\
	Then 
	\\ \\
	\begin{equation}	
	\begin{gathered}
	t=\frac{ \hat{\beta_{1}} - m } { s(\hat{\beta_{1}}) }
	\end{gathered}
	\end{equation}			
	\\ \\
	Is distributed as a Student's t with degrees of freedom as determined by the estimation of $\sigma^2$.
	\\ \\
	In similar manner, t-test of hypothesis about $\beta_{0}$ and any of the $\hat {Y_{i}}$ can be constructed. In each case, the denominator 
	is the standard deviation of the estimate:
	
	\begin{equation}
	Var(\hat{\beta_{1}}) = s^2 / \sum x_{i}^2
	\end{equation}		
	
	\begin{equation}
	Var(\hat{\beta_{0}}) = s^2 ( (1/n) + \overline{X}^2 / \sum{x_{i}^2)
	\end{equation}			
	
	The MS(Regr) is an estimate of $\sigma^2+ \beta_{1} \sum x_{i}^2$ and that MS(Res) is an estimate of $\sigma^2$, thus, they are two 	
	independent $\chi^2$ distributions with one and two degrees of freedom. If the null hypothesis $\beta_{1}$=0 is true, both MS(Regr) and MS(
	Res) are estimating $\sigma^2$, 
	therefore, the ratio: 
	
	\begin{equation}
	F=MS(Regr) / MS(Res)
	\end{equation}	
	
	F is F(1,n-2), an F distribution with 1 and n-2 degrees of freedom and can be used as an alternative to the t-test for two tailed 
	hypotheses about the regression coefficients. 
	\\ \\
	
	The confidence interval estimates of parameters are more informative than point estimates because they reflect the precision of the estimates. The 95\% confidence interval estimate of $\beta_{1}$ and $\beta_{0}$ are respectively:
	
	\begin{equation}
	\hat{\beta_{1}} \pm t_{(0.025, \nu)} s (\hat{\beta_{1}})
	\end{equation}		
	
	and
	
	\begin{equation}
	\hat{\beta_{0}} \pm t_{(0.025, \nu)} s (\hat{\beta_{0}})
	\end{equation}		
		
			where $\nu$ is the degrees of freedom associated with s^2.
			
\section{VIOLATION OF ASSUMPTIONS}

	\begin{itemize}
		\item The Normality assumption is needed for tests of significance and construction of the confidence intervals
		\item If Cov($e_{i}, e_{j}) \neq$ 0 for a pair of i, j, the errors are not independent and the least square estimators continue to be 		
					unbiased but are no longer the best estimators
		\item The effect of non-constant (heterogeneous) variance will violate the derivation of the estimators of the variances of $\hat{\beta_{
					0}}$ and $\hat{\beta_{1}}$.	
	\end{itemize}


\section{EXAMPLE: Dat1}
	
	The data Dat1 has two columns: one is the out column as a numeric variable tht we will use as our dependent variable and a column 
	exp, a variable with two categories that we recoded as numeric variable in $exp_n$, this variable will be our independent variable 
	measured without errors.
	
	A summary of the these two variables is this \textbf{Table 1}:
	
		\begin{figure}[htbp]
		\centering
			\includegraphics[scale=1.5]{SAS}
	\end{figure}
	
\newpage
	
	Our purpose is to use the summary and formulas from the book and derive estimates for model (1), that is, we will calculate $\beta_{0}, 		
	\beta_{1}$, sum of squares, variances and t-tests for the $\beta$ coefficients, and the adjusted R^2:
	\\ \\
	Compare these results $\textbf{Calculations by Hand}$ vs. estimates using $\textbf{lm}$ in R:

	\begin{figure}[htbp]
		\centering
			\includegraphics[scale=1.5]{RA}
	\end{figure}
	
	\begin{figure}[htbp]
		\centering
			\includegraphics[scale=1.5]{RB}
	\end{figure}
	
	\begin{figure}[htbp]
		\centering
			\includegraphics[scale=1.5]{RC}
	\end{figure}
		
	\begin{figure}[htbp]
		\centering
			\includegraphics[scale=1.5]{RD}
	\end{figure}
			
\newpage


%%%%%%%%%%%%%%%%%%%%%%%%%%%%%%%%%%%%%%%%%%%%%%%%%%%%%%%%%%%%%%%%%%%%%%%%%%%%%%%%%%%%%%%%%%%%%%%%%%%%%%%%%%%%%%%%%%%%%%%%%%%%%%%%%%%

\end{document} 
%%%%%%%%%%%%%%%%%%%%%%%%%%%%%%%%%%%%%%%%%%%%%%%%%%%%%%%%%%%%%%%%%%%%%%%%%%%%%%%%%%%%%%%%%%%%%%%%%%%%%%%%%%%%%%%%%%%%%%%%%%%%%%%%%%%
